\documentclass{article}

% If you're new to LaTeX, here's some short tutorials:
% https://www.overleaf.com/learn/latex/Learn_LaTeX_in_30_minutes
% https://en.wikibooks.org/wiki/LaTeX/Basics

% Formatting
\usepackage[utf8]{inputenc}
\usepackage[margin=1in]{geometry}
\usepackage[titletoc,title]{appendix}

% Math
% https://www.overleaf.com/learn/latex/Mathematical_expressions
% https://en.wikibooks.org/wiki/LaTeX/Mathematics
\usepackage{amsmath,amsfonts,amssymb,mathtools}

% Images
% https://www.overleaf.com/learn/latex/Inserting_Images
% https://en.wikibooks.org/wiki/LaTeX/Floats,_Figures_and_Captions
\usepackage{graphicx,float}

% Tables
% https://www.overleaf.com/learn/latex/Tables
% https://en.wikibooks.org/wiki/LaTeX/Tables

% Algorithms
% https://www.overleaf.com/learn/latex/algorithms
% https://en.wikibooks.org/wiki/LaTeX/Algorithms
\usepackage[ruled,vlined]{algorithm2e}
\usepackage{algorithmic}

% Code syntax highlighting
% https://www.overleaf.com/learn/latex/Code_Highlighting_with_minted
\usepackage{minted}
\usemintedstyle{borland}

% References
% https://www.overleaf.com/learn/latex/Bibliography_management_in_LaTeX
% https://en.wikibooks.org/wiki/LaTeX/Bibliography_Management
\usepackage{biblatex}
\addbibresource{references.bib}

% Title content
\title{Assignment 1 STA}
\author{Abhijay Mitra, 18IE10001}
\date{January 26, 2021}

\begin{document}

\maketitle

% Q 4 starts here
\section{Knapsack algorithm}
\subsection{Problem Statement}
There are $n$ items with values $v_1,\ldots,v_n$, and sizes $s_1,\ldots,s_n$.
Consider the following greedy algorithm for the knapsack problem.
We initially sort all the items in order of non-increasing ratio of
value to size so that $v_1/s_1 \geq v_2/s_2 \geq \ldots \geq v_n/s_n$.
Let $i^*$ be the index of an item of maximum value so that
$v_{i^*} = \max\limits_{i \in I} v_i$.
The greedy algorithm puts items in the knapsack in index order
until the next item no longer fits;
that is, it finds $k$ such that
$\sum\limits_{i=1}^k s_i \leq B$ and
$\sum\limits_{1 \leq i \leq k+1} s_i > B$, where $B = $ knapsack capacity.
The algorithm returns either $\{1, \ldots, k\}$ or $\{i^*\}$,
whichever has greater value.
Prove that this algorithm is a 1/2-approximation algorithm for the knapsack problem.

% Q4 solution starts here
\subsection{Solution}
This is a greedy algorithm and works fine for fractional knapsack problem. In the Solution to the fractional knapsack problem version of this statement, we would be choosing the first k elements completely and the $i^{* th}$ element partially. \newline
Let's assume the optimal solution of the fractional knapsack problem be $OPT_f$ = $\sum_{i \in \{1,\ldots, k\}}{v_i}$ + $\alpha v_{i^*}$ where $0 \leq \alpha \leq 1$. \newline
Our algorithm is supposed to choose the maximum of $\sum_{i \in \{1,\ldots, k\}}{v_i}$ and $\alpha v_{i^*}$. as mentioned in the problem statement. So, if the approx cost returned to us by our greedy algorithm to our ${0 - 1}$ knapsack problem is $X$, we can say \newline
\linespread{1.6}$X \hspace{5} = \hspace{5} max((\sum_{i \in \{1,\ldots, k\}}{v_i}), v_{i^*})\hspace{5}$ \geq $\hspace{5} max(\sum_{i \in \{1,\ldots, k\}}{v_i}, \alpha v_{i^*}) \hspace{5} \geq \hspace{5}{\sum_{i \in \{1,\ldots, k\}}{v_i} + \alpha v_{i^*}} \over 2 \hspace{5}$ $\hspace{5} \geq {{OPT_f} \over 2} \hspace{10}$\{as $ 0 \leq \alpha \leq 1\}$  \newline \linespread{1.6}
Also we know that fractional knapsack outputs sum of costs at-least the same as the $0 - 1$ knapsack we can say,
${{OPT_f}} \geq X$ and we have proven that $2 X \geq $ ${{OPT_f}} $.\newline Combining both results, we have${{OPT_f} \over 2} \leq X \leq {{OPT_f}}$making $X$ a $1 \over 2$ approximate solution.

% Q3 starts
\section{Max-cut}
\subsection{Problem Statement}
Let $G = (V,E)$ be an undirected graph. Let $(A,B)$ be a cut of $V$.
Let $v \in V$. Then, either $v \in A$ or $v \in B$, but not both.
If $v \in A$, then we define $d_{in}(v) = |\{(v,u) \in E \mbox{ for which }u\in A\}|$ and $d_{out}(v) = |\{(v,u) \in E \mbox{ for which }u \in B\}|$.
If $v \in B$, then we define $d_{in}(v) = |\{(v,u) \in E \mbox{ for which }u\in B\}|$ and $d_{out}(v) = |\{(v,u) \in E \mbox{ for which }u \in A\}|$.
A vertex $v \in V$ qualifies for a \emph{flip} if $d_{out}(v) < d_{in}(v)$.
The algorithm is given below:

\begin{algorithm}
\caption{Max-cut via vertex flipping}
 Let $(A,B)$ be an arbitrary cut of $V$\\
 \While{there exists a vertex $v \in V$ which qualifies for a flip}{
   \If{$v \in A$}{
     $A = A \smallsetminus \{v\}$\\
     $B = B \cup \{v\}$\\}
   \Else{
     $B = B \smallsetminus \{v\}$\\
     $A = A \cup \{v\}$\\}}
 \Return the cut $(A,B)$
\end{algorithm}

Show that the algorithm needs poly time and the approximation ratio is 1/2.
\subsection{Solution}
Add your introduction and overview here.

% Example Subsection
\subsection{Subsection Title}
This is a subsection.

% Example Subsubsection
\subsubsection{Subsubsection Title}
This is a subsubsection.

%  Theoretical Background
\section{Theoretical Background}
Add your theoretical background here. Some example text: As we learned from our textbook \cite{kutz_2013}, Fourier introduced the concept of representing a given function $f(x)$ by a trigonometric series of sines and cosines:
\begin{equation}
    f(x) = \frac{a_0}{2} + \sum_{i=1}^\infty \left(a_n\cos{nx} + b_n\sin{nx}\right) \quad x \in (-\pi,\pi].
    \label{eqn:fourierseries}
\end{equation}
You can reference numbered equations, figures, tables, algorithms, and code like this: Equation~\ref{eqn:fourierseries}, etc.

% Algorithm Implementation and Development
\section{Algorithm Implementation and Development}
Add your algorithm implementation and development here. See Algorithm~\ref{alg:example} for how to include an algorithm in your document. This is how to make an \textit{ordered} list:
\begin{enumerate}
    \item Fluffy swallowed a marble.
    \item I took Fluffy to the vet.
    \item They took an ultrasound of Fluffy's intestines.
\end{enumerate}

\begin{algorithm}
\begin{algorithmic}
    \STATE{Import data from \texttt{Testdata.mat}}
    \FOR{$j = 1:20$}
        \STATE{Extract measurement $j$ from \texttt{Undata}}
        \STATE{Do something useful}
    \ENDFOR
    \IF{$i\geq 5$}
        \STATE{$i\gets i-1$}
    \ELSE
        \IF{$i\leq 3$}
            \STATE{$i\gets i+2$}
        \ENDIF
    \ENDIF
\end{algorithmic}
\caption{Example Algorithm}
\label{alg:example}
\end{algorithm}

% Computational Results
\section{Computational Results}
Add your computational results here. See Table~\ref{tab:mascots} for how to include a table in your document. See Figure~\ref{fig:dubs} for how to include figures in your document.

\begin{table}
    \centering
    \begin{tabular}{rll}
    & Name & Years \\
    \hline
    1 & Frosty & 1922-1930  \\
    2 & Frosty II & 1930-1936 \\
    3 & Wasky & 1946 \\
    4 & Wasky II & 1947 \\
    5 & Ski & 1954 \\
    6 & Denali & 1958 \\
    7 & King Chinook & 1959-1968\\
    8 & Regent Denali & 1969 \\
    9 & Sundodger Denali & 1981-1992 \\
    10 & King Redoubt & 1992-1998 \\
    11 & Prince Redoubt & 1998 \\
    12 & Spirit & 1999-2008 \\
    13 & Dubs I & 2009-2018 \\
    14 & Dubs II & 2018-Present
    \end{tabular}
    \caption{UW mascots as described in \cite{washington_huskies}.}
    \label{tab:mascots}
\end{table}

% begin{figure}[tb] % t = top, b = bottom, etc.
\begin{figure}
    \centering
    \includegraphics[width=0.5\linewidth]{dubs.jpg}
    \caption{Here is a picture of Dubs \cite{webeck_2018}. Dubs did not swallow a marble.}
    \label{fig:dubs}
\end{figure}

% Summary and Conclusions
\section{Summary and Conclusions}
Add your summary and conclusions here.

% References
\printbibliography

% Appendices
\begin{appendices}

% MATLAB Functions
\section{MATLAB Functions}
Add your important MATLAB functions here with a brief implementation explanation. This is how to make an \textbf{unordered} list:
\begin{itemize}
    \item \texttt{y = linspace(x1,x2,n)} returns a row vector of \texttt{n} evenly spaced points between \texttt{x1} and \texttt{x2}.
    \item \texttt{[X,Y] = meshgrid(x,y)} returns 2-D grid coordinates based on the coordinates contained in the vectors \texttt{x} and \texttt{y}. \text{X} is a matrix where each row is a copy of \texttt{x}, and \texttt{Y} is a matrix where each column is a copy of \texttt{y}. The grid represented by the coordinates \texttt{X} and \texttt{Y} has \texttt{length(y)} rows and \texttt{length(x)} columns.
\end{itemize}

% MATLAB Codes
\section{MATLAB Code}
Add your MATLAB code here. This section will not be included in your page limit of six pages.

\begin{listing}[h]
\inputminted{matlab}{example.m}
\caption{Example code from external file.}
\label{listing:examplecode}
\end{listing}

\end{appendices}

\end{document}
